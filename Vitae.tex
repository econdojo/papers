%------------%
%  Preamble  %
%------------%

\documentclass[final,11pt]{article}
\usepackage[paperwidth=9in,top=1.0in, bottom=1.0in, left=1.1in, right=1.1in]{geometry}
\usepackage{color}
\usepackage{multirow}
\usepackage{setspace}
\usepackage{fancyhdr}
\usepackage{longtable}
\usepackage{array}
\usepackage{booktabs}
\usepackage{mathpazo}
\usepackage[colorlinks, linkcolor=blue, anchorcolor=blue, citecolor=blue]{hyperref}

\renewcommand{\headrulewidth}{0pt}
\newcommand{\paperlist}[1]{\hspace*{1cm}\begin{minipage}[t]{.95\textwidth}\setlength{\parindent}{-0.5cm}{#1}\end{minipage}}
\newcommand{\reflist}[1]{\hspace*{0.5cm}\begin{minipage}[t]{0.98\textwidth}\setlength{\parindent}{0cm}{#1}\end{minipage}}
\setlength{\arraycolsep}{10pt}
\setlength\topmargin{-1.5cm}
\setlength\headheight{0.5cm}
\setlength\headsep{0.8cm}
\setlength\footskip{1.0cm}
\setlength{\parindent}{0em}
\pagestyle{fancy}
\cfoot{\textcolor[rgb]{0.5,0.5,0.5}{\sc Fei Tan: \today}}
\rfoot{\thepage\qquad\qquad}

%------------%
%  Document  %
%------------%

\begin{document}
\begin{spacing}{1.20}

\begin{center}
    {\bf \Large FEI\ \ TAN}\\[5pt]
    {\bf \large \sc Curriculum Vitae}\\[5pt]
\end{center}

\begin{tabular}[t]{l@{\hspace{1.0cm}} l@{\hspace{2.0cm}} l}
    {\sc Contact} & Chaifetz School of Business & Office: Davis-Shaughnessy Hall 469A \\
    {\sc Information} & Saint Louis University & Discord: \href{https://discord.gg/SsrNPEeP2P}{\color{blue}discord.gg/SsrNPEeP2P} \\
    & 3674 Lindell Boulevard & E-mail: \href{mailto:econdojo@gmail.com}{\color{blue}econdojo@gmail.com} \\
    & St. Louis, MO 63108-3397 & Homepage: \href{https://github.com/econdojo}{\color{blue}github.com/econdojo} \\[5pt]
\end{tabular}

\begin{center}
    {\bf PROFESSIONAL POSITIONS}\\[5pt]
\end{center}

\begin{tabular}[t]{@{\hspace{.8cm}} l}
    Associate Professor of Economics, Saint Louis University, 2022 - \\
    Assistant Professor of Economics, Saint Louis University, 2015 - 2022 \\
    Research Fellow, Zhejiang University of Finance and Economics, 2015 - \\[5pt]
\end{tabular}

\begin{center}
    {\bf EDUCATIONAL BACKGROUND}\\[5pt]
\end{center}

\begin{tabular}[t]{@{\hspace{.8cm}} l}
    Indiana University, Ph.D. in Economics, Bloomington, IN, Aug. 2009 - Jun. 2015 \\
    Indiana University, M.A. in Mathematics, Bloomington, IN, Aug. 2010 - Feb. 2015 \\
    Zhejiang University, B.A. in Economics, Hangzhou, China, Sep. 2005 - Jun. 2009 \\[5pt]
\end{tabular}

\begin{center}
    {\bf FIELDS OF RESEARCH INTEREST}\\[5pt]
\end{center}

\hspace{0.8cm}Machine Learning, Game Theory, Macroeconomics, Bayesian Statistics, Space Economy \\[5pt]

\begin{center}
    {\bf RESEARCH ARTICLES}\\[5pt]
\end{center}

{\sc \underline{Machine Learning}}\\

\paperlist{-- (With Siddhartha Chib) Learning the Macroeconomic Language, manuscript, 2025.}\\[5pt]

{\sc \underline{Game Theory}}\\

\paperlist{-- (With Leifei Lyu) Forecasting the Forecasts of Others on Social Networks, manuscript, February 2025.}\\

\paperlist{-- (With Jieran Wu) Analytic Policy Function Iteration, \textit{Journal of Economic Theory}, Volume 200, March 2022.}\\

\paperlist{-- (With Todd B. Walker) Solving Generalized Multivariate Linear Rational Expectations Models, \textit{Journal of Economic Dynamics and Control}, Volume 60, November 2015, Page 95--111.}\\

\paperlist{-- (With Hang Ye, Shu Chen, Jun Luo, Yongmin Jia, and Yefeng Chen) Increasing Returns to Scale: The Solution to the Second-Order Social Dilemma, \textit{Scientific Reports} 6, Article No. 31927, August 2016.}\\

\paperlist{-- (With Hang Ye, Mei Ding, Yongmin Jia and Yefeng Chen) Sympathy and Punishment: Evolution of Cooperation in Public Goods Game, {\it Journal of Artificial Societies and Social Simulation}, Volume 14, Issue 4, October 2011.}\\[5pt]

{\sc \underline{Macroeconomics}}\\

\paperlist{-- (With Leifei Lyu and Zheliang Zhu) Policy Rule Regressions with Survey Data, revised and resubmitted at \textit{Journal of Economic Dynamics and Control}, October 2025.}\\

\paperlist{-- (With Yunpeng Fan and Zheliang Zhu) The Information Choice of Monetary Policy, manuscript, August 2025.}\\

\paperlist{-- (With Jieran Wu) The Information Channel of Monetary and Fiscal Policy, manuscript, August 2024.}\\

\paperlist{-- Appetite for Treasuries, Debt Cycles, and Fiscal Inflation, revision requested at \textit{Macroeconomic Dynamics}, August 2024.}\\

\paperlist{-- (With Yoosoon Chang and Junior Maih) Origins of Monetary Policy Shifts: A New Approach to Regime Switching in DSGE Models, \textit{Journal of Economic Dynamics and Control}, Volume 133, December 2021.}\\

\paperlist{-- A Frequency-Domain Approach to Dynamic Macroeconomic Models, \textit{Macroeconomic Dynamics}, Volume 25, Issue 6, September 2021, Page 1381--1411.}\\

\paperlist{-- (With Bing Li and Pei Pei) Financial Distress and Fiscal Inflation, \textit{Journal of Macroeconomics}, Volume 70, December 2021.}\\

\paperlist{-- An Analytical Approach to New Keynesian Models under the Fiscal Theory, \textit{Economics Letters}, Volume 156, July 2017, Page 133--137.}\\

\paperlist{-- Interpreting Rational Expectations Econometrics via Analytic Function Approach, \textit{Economics Bulletin}, Volume 37, Issue 2, June 2017, Page 1182--1190.}\\[5pt]

{\sc \underline{Bayesian Statistics}}\\

\paperlist{-- (With Siddhartha Chib and Minchul Shin) DSGE-SVt: An Econometric Toolkit for High-Dimensional DSGE Models with SV and $t$ Errors, \textit{Computational Economics}, October 2021.}\\

\paperlist{-- (With David E. Rapach) Bayesian Estimation of Macro-Finance DSGE Models with Stochastic Volatility, revision requested at \textit{Journal of Applied Econometrics}, June 2020.}\\[5pt]

{\sc \underline{Space Economy}}\\

\paperlist{-- (With Xiao Zhang) The Space-AI Convergence: Why Every Company Needs an Orbital Strategy, manuscript, December 2025.}\\[5pt]

\begin{center}
    {\bf PROFESSIONAL EXPERIENCE}\\[5pt]
\end{center}

{\sc \underline{Teaching Experience}}\\

\paperlist{Graduate: Neural Networks (SLU), Bayesian Statistics (SLU), Math Camp for Ph.D. Students (IU)}\\

\paperlist{Undergraduate: Game Theory (SLU), Intermediate Macroeconomics (SLU, IU), Introduction to Microeconomics \& Macroeconomics (IU), International Macroeconomics (SLU), Money and Banking (SLU, IU)}\\[5pt]

{\sc \underline{Conferences and Seminars}}\\

\paperlist{2021 - 25: NBER-NSF Seminar on Bayesian Inference in Econometrics and Statistics (Washington University in St. Louis), Midwest Econometrics Group Conference (Michigan State University, University of Kentucky), Society for Nonlinear Dynamics and Econometrics (University of Central Florida), International Association for Applied Econometrics (Xiamen University), Chinese Economists Society (Sun Yat-sen University)}\\

\paperlist{2016 - 20: NBER-NSF Seminar on Bayesian Inference in Econometrics and Statistics (Washington University in St. Louis), Midwest Econometrics Group Meeting (St. Louis Fed, Texas A\&M University, University of Wisconsin-Madison, Dallas Fed), Midwest Macroeconomics Conference (University of Pittsburgh), China Meeting of the Econometric Society (Fudan University)}\\

\paperlist{2011 - 15: Saint Louis University, Texas Tech University, Missouri Economics Conference (University of Missouri), Midwest Econometrics Group Meeting (Indiana University), Midwest Macroeconomics Conference (University of Colorado), Washington University Grad Student Conference, Jordan River Economics Conference (Indiana University)}\\[5pt]

{\sc \underline{Other Activities and Services}}\\

\paperlist{SLU: aiTECH Tiger Team (2025), Graduate Board Committee (2016, 2019 - 2021, 2025), Research and Scholarship Committee (2025), Diversity Committee (2024), Simon Endowed Chair Search Committee (2023), DEI Director Search Committee (2021), Faculty Senate (2020 - 2024), Research Seminar Organizer (2017 - 2021), Math Education Committee (2016 - 2020)}\\[5pt]

{\sc \underline{Referee Experience}}\\

\reflist{European Economic Review, Journal of Econometrics (x2), Journal of Economic Dynamics and Control (x2), Economics Letters (x2), Computational Economics, Macroeconomic Dynamics, Journal of Macroeconomics, Journal of Difference Equations and Applications, Energy Economics}\\[5pt]

{\sc \underline{Doctoral Students and Placement}}\\

\paperlist{WUSTL: Leifei Lyu (Central University of Finance and Economics, 2025), Li Zhang (Navy Federal Credit Union, 2024), Zheliang Zhu (Southwestern University of Finance and Economics, 2023), Yi-chun Lin (Analysis Group, 2023)}\\[5pt]

\begin{center}
    {\bf AWARDS AND HONORS}\\[5pt]
\end{center}

\begin{longtable}[t]{@{\hspace{-1.5cm}} l}
    Three-Year Research Award, Chaifetz School of Business, Saint Louis University, 2023 \\
    Summer Grant Award, Chaifetz School of Business, Saint Louis University, 2019 - 2022 \\
    Annual Research Award, Chaifetz School of Business, Saint Louis University, 2016, 2022 \\
    Henry M. Oliver Award for Excellence in Economic Theory, Indiana University, 2014 \\
    Best Third Year Paper Award, Department of Economics, Indiana University, 2012 \\
    Best Graduation Thesis, College of Economics, Zhejiang University, 2009 \\
    First Prize of 9th Challenge Cup Thesis Oral Defense, Zhejiang University, 2008 \\
    Chu Kochen Honors College Students, Zhejiang University, 2005\\[5pt]
\end{longtable}

\begin{center}
    {\bf PERSONAL INFORMATION}\\[5pt]
\end{center}

\begin{longtable}[t]{@{\hspace{-3.5cm}} l}
    Background: China citizenship, U.S. permanent resident, single with one cat \\
    Languages: Chinese, English, Python, MATLAB, C, C++ \\
    Hobbies: programming, trading, soccer, basketball, swimming, table tennis\\[5pt]
\end{longtable}

\begin{center}
    {\bf BIOGRAPHY}\\[5pt]
\end{center}

Grounded in first principles, Dr. Fei Tan's research develops algorithmic solutions for complex social systems at the intersection of artificial intelligence, economics, and statistics. One line of his research develops time-domain and frequency-domain approaches to dynamic equilibrium models with strategic agents. Another line develops Bayesian methods for estimating large-scale structural models. These tools are applied to study economic policy, asset prices, and social networks. Recently, he likes to train neural nets regularized by economic theory. His previous research studies the evolution of cooperative and altruistic human behavior.\\

As an educator, Dr. Tan teaches economics, statistics, and computer science with comprehensive open-source curricula on GitHub. He also co-founded \href{https://www.youtube.com/@BusinessSchool101}{\color{blue}Business School 101}, a YouTube channel with over 100k subscribers democratizing business education. Beyond the classroom, he is a thematic investor for space economy and has rich experience in derivative trading.\\

Dr. Tan earned his B.A. in Economics from Zhejiang University and both M.A. in Mathematics and Ph.D. in Economics from Indiana University. He currently serves as Associate Professor of Economics at Saint Louis University, where he leads the aiTECH Tiger Team to integrate state-of-the-art AI technologies into teaching and research at higher education institutions.

\end{spacing}
\end{document}
